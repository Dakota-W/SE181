\documentclass[10pt]{article}
\usepackage{enumitem}
\usepackage{listings}
\usepackage{underscore}
\usepackage[ddmmyyyy]{datetime}
\renewcommand{\dateseparator}{--}
\usepackage[bookmarks=true]{hyperref}
\usepackage[utf8]{inputenc}
\usepackage[english]{babel}
\usepackage{url}
\usepackage{graphicx}

\newenvironment{enum}
{\begin{enumerate}[label*=\arabic*.][resume]}
{\end{enumerate}}

\hypersetup{
    pdftitle={Software Requirement Specification},    
    pdfauthor={Dakota Wessel},                     
    pdfsubject={TeX and LaTeX},                        
    pdfkeywords={TeX, LaTeX, graphics, images}, 
    colorlinks=true,       
    linkcolor=blue,       
    citecolor=black,       
    filecolor=black,        
    urlcolor=purple,        
    linktoc=page            
}
\def\myversion{1.0.1}
\date{}
\usepackage{hyperref}
\begin{document}

\begin{flushright}
    \rule{16cm}{5pt}\vskip1cm
    \begin{bfseries}
        \Huge{SOFTWARE REQUIREMENTS\\ SPECIFICATION}\\
        \vspace{1.0cm}
        for\\
        \vspace{1.0cm}
        Checkers\\
        \vspace{1.5cm}
        \LARGE{Version \myversion}\\
        \vspace{1.5cm}
        Prepared by:\\
    Adam Luong\\
    Benny Mai\\
    Dakota Wessel\\
    Jacky Zheng\\
    Tony Zhu\\
        \vspace{1.9cm}
        Group: \textbf{10}\\
        \vspace{1cm}
        \today\\
    \end{bfseries}
\end{flushright}

\tableofcontents

\section*{Revision History}

\begin{center}
    \begin{tabular}{|c|c|c|c|}
        \hline
        Name & Date & Reason For Changes & Version\\
        \hline
        1.0.0 & \formatdate{15}{10}{20} & Initial Setup & Dakota Wessel\\
        \hline
        1.0.1 & \formatdate{18}{10}{20} & Intro. and Use Cases & Benny Mai\\
        \hline
    \end{tabular}
\end{center}

\section{Introduction}

\subsection{Purpose of Document}
The goal of this document is to outline the requirement specifications of our web-based checkers game. 
This application will allow two users to connect and interact remotely, allowing them to play and chat.
The application will follow the standard U.S. rules of American checkers. 
This document will cover the scope, objective, basic requirements, and goals for this application. 
After the application's high-level look, this document will dive deeper into topics like functional, 
non-functional, user interface, design, test cases, program usage, and references. 
This document will clearly explain to an engineer or end-user the overall implementation 
and goals of our application. The final version of the game will be playable through the network. Therefore, 
two players on the web browser will be able to play against each other.

\subsection{Project Scope}
The main objective for documentation is to educate the reader about our Checkers application, 
its functionality, the technologies used, and outline the application requirements.
The document will capture the basic concept of the game with its functional and non-functional requirements. 
Each player’s inputs will follow a guideline of possible use cases that is linked with the GUI as well providing a description of each.

\subsection{Overview of Document}
The documentation will provide a clear explanation about which technologies we used, 
how we implemented them, and why we chose to use them. It will outline each component 
of our application. The flow starts with the functional requirements, non-functional requirements, 
user interface, and finished with lobbies, gameplay, and winning conditions, concluding with our references.
The document will contain diagrams of a mock-up workflow of the possible use cases and a navigation flow, which may be changed throughout the process. 

\subsection{Background}

\subsubsection{History}

Throughout history, the game Checkers has been around, so the exact date for Checkers' 
invention is unknown. One of the earliest records of the game dates back to 3000 B.C in what 
is present-day Iraq. Later in Egypt, in 1400 B.C, the game was played using a 5 x 5 board \cite{historyCheckers}. 
However, the version of Checkers that we know of today was established in the mid-1500s by an English mathematician. 
Now the board game of checkers is cemented as one of the most popular board games of all time. 
\subsubsection{Game Rules}
    The rules provided will follow the American Checker Federation and Standard U.S. rules. \cite {checkersFoundation}.
    \begin{enumerate}
    \item The first turn will be decided by a randomized generator.
    \item A player can forfeit at any time; as a result, the opponent wins.
    \item Each player is given 5 minutes to make a move. If the player fails to move within the time limit, the player that failed to move loses the game.
    \item A player can request a draw anytime. If a player requests a draw, and the opposing also requests it, the game will result in a draw. The draw request doesn’t have a time limit. It’ll act as an on and off switch, but each player will be notified of a request for a draw.
    \end{enumerate}
\subsubsection{Moves} 
\begin{enumerate}
    \item A player may only move their own pieces.
    \item Normal Piece
        \subitem A normal piece may only move toward the other player's side of the board.
        \subitem A normal piece may move diagonally to the left or right to a vacant square in front of it.
        \subitem A normal piece may capture on the diagonal if there exists a vacant square one more diagonal position ahead.
            \subsubitem A piece may move again if there exists another piece to capture after making a capture.
    \item A King Piece moves the same as a normal piece but can move and capture backward.
    \item A King Piece may capture forward or backward.
    \item If a normal piece reaches the opposite edge of the board, it becomes a King Piece.
\end{enumerate}   

\subsubsection{Win Condition}
\begin{enumerate}
    \item When one player has no more pieces to move, the other player is the winner.
    \item If a player forfeits the match, the other player is conceded the winner.
    \item If both players request a draw, the match will result in a tie.
    \item If a player fails to make a turn before the timer ends, they lose.
    \item If a player puts the opposing player in a position where they cannot make a legal move for any pieces in the following turn, player A wins.
    \item If a player captures all the opposing player pieces, they win.
\end{enumerate}

\subsection{Abstract}

Our goal is to create a document that will help a team member create an application, 
both client-side and server-side, that allows remote users to play checkers. The construction of this application will host one game of checkers to two users. 
\section{Overall Description}

\subsection{Product Functions}

\begin{enumerate}
    \item Provide an application to host a checkers game with two user over the local network or internet.
    \item Provide a server that mediates gameplay, game sessions, and client interactions.
\end{enumerate}

\subsection{Assumptions and Dependencies}

\begin{enumerate}
    \item A connection to a local network or internet connection.
    \item A computer with a graphical environment for the client
    \item A Unix or Windows based server.
    \item Knowledge of the rules of checkers.
    \item Client know how to launch a python file.
\end{enumerate}

\section{Functional Requirements}

\subsection{Client}

\begin{enumerate}[label*=R\arabic*.]
    \item Client - Server Interaction
    \begin{enumerate}[label*=\arabic*.]
        \item Client will automatically check to see if the game server is live given the set URL in the program.
        \item Client will not be able to join a lobby if no response is sent from the requested server URL.
        \item Client will be able to terminate itself from the server at any given moment.
        \item Client will be able to reconnect back to the lobby within 30 seconds, else they will receive a notification saying that the lobby has been closed and they are unable to rejoin due to the timeout period.
        \item Client should be able to distinguish themselves with a valid player name when they join a lobby.
    \end{enumerate}
\end{enumerate}

\subsubsection{Board State}

\begin{enumerate}[resume*]
    \item Board
    \begin{enumerate}[label*=\arabic*.]
        \item The clients will have a copy of the board for rendering purposes
        \item The board will update upon receiving a server update
    \end{enumerate}
\end{enumerate}

\subsection{Server}

\begin{enumerate}[resume*]
    \item Server should be able to be run constantly without crashing.
        \subitem Server will have a heartbeat function that will send an email to the developers if the server is down.
    \item Server - Client Interaction
    \begin{enumerate}[label*=\arabic*.]
        \item Server will be able to process client connection information to create a lobby.
        \item Server will keep track of clients and game sessions.
        \item On user timeout, wait a set amount of time.
            \subitem If not connected within that time, signal other client game has ended due to disconnected player.
        \item Server will validate moves before sending updated move to clients.
            \subitem On invalid move, signal client that move was invalid and to try again.
    \end{enumerate}
    \item Server - Lobby Interaction
    \begin{enumerate}[label*=\arabic*.]
        \item Server will first check that a port is open before assigning it to a lobby upon creation.
        \item Server will close the port assigned to a lobby when the game has been finished.
    \end{enumerate}
\end{enumerate}

\section{Other Requirements}

\subsection{System Requirements}

\begin{enumerate}[label*=S\arabic*.]
    \item Server and Client
        \subitem Python3
    \item Client
        \subitem Windowing display environment:
            \subsubitem Windows
            \subsubitem MacOS
\end{enumerate}

\subsection{Network Requirements}

\begin{enumerate}[label*=N\arabic*.]
    \item Client and Server
	 \begin{enumerate}[label*=\arabic*.]
        \item An active internet connection
        \item port forwarding configured properly on their local network
        \item Client must be connected to Drexel's network
        \item Response time to the server must be less than 120ms
	  \end{enumerate}
    \item Server
 	\begin{enumerate}[label*=\arabic*.]
        \item Server must be hosted on tux.cci.drexel.edu
        \item Server will be running on one dedicated box
  \end{enumerate}
\end{enumerate}

\section{User Interface}

\subsection{Framework}

The project shall use Python3? to create the user interface.

\subsection{Menus}

\begin{itemize}
\item Main Menu
    \subitem Play Game button: connects player to the server
    \subitem Quit Game: closes the game

\item Game Lobby
    \subitem Show players in the lobby
    \subitem Ready? button: start game once both players click
    \subitem Main Menu button: return to the main menu

\item Game
    \subitem Checkers match screen, showing a checkers board with both players’ pieces
        \subsubitem Player’s “side” of the board is always on the bottom of the screen
    \subitem Top of screen shows whose turn it is: Black or Red
        \subsubitem During the player’s turn, they click on a piece to select it
        \subsubitem Piece becomes highlighted, as does all valid spaces the player could move to
        \subsubitem Player clicks on a highlighted space to move the piece there or clicks on a new piece to select it
        \subsubitem After a move has been made, the game checks if there is a winner. Move to Winner Display if a winner is found. Otherwise, start next player’s turn.

\item Winner Display
    \subitem Show the player who won the game: Black or Red
    \subitem Rematch? No button: return both players to the main menu
    \subitem Rematch? Yes button: register this player as wanting a rematch
        \subsubitem If both players click Yes, then return both players to the Game Lobby screen
\end{itemize}

\section{Standard Components}

\begin{itemize}
    \item Buttons: used for menu interations and navigation.
    \item Pieces: The pieces will be the basic playing pieces within gameplay.
    \item King Pieces: These pieces will behave identically to the regular pieces, but with additional movement options according to the rules of Checkers.
    \item Checkers Board: an 8x8 grid of alternating black and red tiles on which all pieces will be displayed in the gameplay.
\end{itemize}

\subsection{Program Usage}
\subsubsection{Gameplay}
\begin{enumerate}[label*=G\arabic*.]
	\item The match begins with an 8x8 grid with each tile alternating between black and red in color.
	\item One player is assigned the black pieces and the other red.
	\item Each player’s pieces start on opposite ends of the board, occupying every other space within the first three rows (for a total of 12 pieces for each player).
	\subitem See attached image for an example checkers setup:
	\subitem \includegraphics[width=8cm]{board.png}
	\item The black player starts the match by taking their turn.
	\item During each turn, the active player selects a piece of their own to move according to the following rules:
\begin{itemize}
    \item If the piece is bordering an enemy piece and there is a free space on the other side of that enemy piece, the piece must “jump” to the empty space, removing the enemy piece it moved over from the board
        \subitem Multiple jumps can be made in a single turn if the piece is in position to jump an additional enemy piece after completing a jump
    \item If the piece cannot jump, it may move diagonally one space to an unoccupied space
        \subitem If the piece is a non-king, it must move forward
        \subitem If the piece is a king, it can move in any direction
    \item A piece cannot jump over pieces of the same color as itself (friendly pieces)
    \item Two pieces cannot occupy the same space, regardless of color
\end{itemize}

\item When a non-king piece has reached the edge of the board opposite its color’s starting side, that piece will be crowned and turned into a king, allowing it to move in any direction.
\end{enumerate}

\subsubsection{Use Case Flow}
\begin{enumerate}
    \item Creating a game
        \subitem Precondition: The user is on the menu screen.
        \subitem Action: The user presses the Create Game button.
        \subitem Postcondition: The user will be led to the board screen where it will contain both non-game related and game elements.
    \item Joining a game
        \subitem Precondition: The user is on the menu screen and will need to be provided a unique identifier from the other player to join the match. The user will also need to be on the menu screen.
        \subitem Action: The user presses the Join Game button in the menu screen.
        \subitem Postcondition: On the menu screen, a text box will pop up prompting the user to type in the unique identifier to join the match.
            \subsubitem If the unique identifier is valid, it will lead the user to the board screen where the host is also located in.
            \subsubitem If the unique identifier is not valid, the user will stay on the menu screen and will be prompted to try again with a valid identifier.
    \item Instructions
        \subitem Precondition: The user is on the menu screen.
        \subitem Action: The user presses the Instruction button.
        \subitem Postcondition: On the menu screen, a text box will pop up providing the rules and how to start/join a game. The user will be provided only one button to close out the instruction text box.
    \item Starting a game
        \subitem Precondition: Both users are on the board screen.
        \subitem Action: Both users press the ready button.
        \subitem Postcondition: A text box pops up in the middle of the screen and randomly generates who goes first and will shortly disappear in a couple of seconds. Then the game starts.
    \item Pause the game (non-game element)
        \subitem Precondition: Both users are on the board screen and the game is started.
        \subitem Action: Both players press the pause button.
        \subitem Postcondition: A textbox for both players is populated given a time until the game is resumed. The timer screen will popup providing the user countdown timer and a button to resume. The screen will disappear only if the timer countdown hits 0 or both players hit the resume button on the timer screen.
    \item Timer for turn (non-game element)
        \subitem Precondition: Both users are on the board screen and the game is started.
        \subitem Action: The timer ends or a player uses the turn before the timer ends.
        \subitem Postcondition: The timer is reset for the next turn.
    \item Requesting a Rematch
        \subitem Precondition: If the board state has reached a draw, defeat, and victory for any player and neither players have left the board screen.
        \subitem Action: Both players are provided a Yes or No button.
        \subitem Postcondition:
            \subsubitem If both players press the Yes Button the match will restart, resetting the board to its initial state as if the game has just been created.
            \subsubitem If one player presses the No Button, both users will return to the menu screen after a short count down.
    \item Moving a checker piece
        \subitem Precondition: It must be a current player’s active turn and the game is started.
        \subitem Action: The player presses a piece that is movable and then presses again for the desired space.
        \subitem Postcondition:
            \subsubitem If it’s an empty space, the user will be allowed to move the desired space otherwise nothing happens. Then the active turn for that user is over while resetting the timer.
            \subsubitem If there’s an enemy piece and there is a diagonal empty space on the enemy’s side, the user will move the empty space. The enemy piece is removed and checks if there is another enemy piece to make an additional jump.
            \subsubitem If a regular piece has reached the edge of the opposite board, it will change to a king piece and be allowed to move in any direction. Either the user will have the option to do another jump if there is another enemy piece or The user can end it’s turn and not commit to the additional jump.
        \subitem Then the active turn for that user is over while resetting the timer.
    \item Regular and King pieces Notes
        \subitem Regular pieces can only make forward single and additional jumps.
        \subitem King pieces can may both forward and backward single and additional jumps.
        \subitem Pieces of the same colors cannot be jumped over.
    \item Completing a match
        \subitem Precondition:
            \subsubitem If either users cannot make any moves on the board screen or have any pieces left.
            \subsubitem If the match gives results to both users a win, lose, or draw.
        \subitem Action: Both users wait temporarily for a textbox to notify the win/lose/draw results to each player.
        \subitem Postcondition: The rematch request is triggered as well as the populating results whether the users win, lose, or resulted in a draw.
\end{enumerate}

\subsubsection{Game Completion Conditions}

\begin{enumerate}
    \item If player A has no more pieces, player B is the winner and vice versa.
    \item If player A disconnects, player B is the winner and vice versa.
    \item If player A forfeits, player B is the winner and vice versa.
    \item If both players request a draw, the game is concluded with a draw.
    \item If player A can no longer make a move in their turn, player B is the winner and vice versa.
    \item If player A can fail to make a move before the timer ends, player B is the winner and vice versa.
\end{enumerate}

\subsubsection{Activity Diagram}
\includegraphics[width=16cm]{UseCaseDiagram.png}


\begin{thebibliography}{9}

\bibitem{checkersFoundation}
  The American Checker Foundation,
  \textit{USA Checkers},
  https://www.usacheckers.com/,
  2019.

\bibitem{historyCheckers}
W.J. Rayment,
\textit{History of Checkers or Draughts},
http://www.indepthinfo.com/checkers/history.shtml,
2004.

\end{thebibliography}


\end{document}